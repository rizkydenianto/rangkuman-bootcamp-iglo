\documentclass{book}
%\documentclass[10pt, a4paper]{article}

% Penyusunan judul dokumen
\title{Buku Catatan Bootcamp}
\author{Rizky Denianto\thanks{Dibuat dengan cinta dan kasih sayang.}}
\date{Desember 2023}

% Paket penulisan kode
\usepackage{listings}
\usepackage{color}

\definecolor{dkgreen}{rgb}{0,0.6,0}
\definecolor{gray}{rgb}{0.5,0.5,0.5}
\definecolor{mauve}{rgb}{0.58,0,0.82}

\lstset{
	frame=tb,
	language=[Sharp]C,
	aboveskip=3mm,
	belowskip=3mm,
	showstringspaces=false,
	columns=flexible,
	basicstyle={\small\ttfamily},
	numbers=none,
	numberstyle=\tiny\color{gray},
	keywordstyle=\color{blue},
	commentstyle=\color{dkgreen},
	stringstyle=\color{mauve},
	breaklines=true,
	breakatwhitespace=true,
	tabsize=3
}

% Ubah tulisan "Contents" ke "Daftar Isi"
\renewcommand{\contentsname}{Daftar Isi}

% Ubah nama "Chapter" menjadi "Bagian"
\renewcommand{\chaptername}{Bagian}

\begin{document}
	% Cetak judul dokumen
	\maketitle
	
	Saya ingin mengucapkan terima kasih kepada diri sendiri yang telah memiliki niat dan usaha dalam penulisan buku ini menggunakan \LaTeX.
	
	% Daftar isi
	\tableofcontents
	
	\chapter{Pelatihan \textit{SQL Server}}
	
	\chapter{Pelatihan .NET}
	
	\section{Pembuatan Proyek}
	\begin{lstlisting}[language=bash]
	dotnet new sln --name <jalur-berkas>
		
	dotnet new console --language "C#" --framework net6.0 --use-program-main --output <jalur-berkas>
	
	dotnet sln <jalur-berkas-.sln> add <jalur-berkas-.csproj>
	\end{lstlisting}
	
	\section{Penulisan Kode C\#}
	
	\subsection{Mencetak string di \textit{Command Line Interface (CLI)}}
	\begin{lstlisting}
	Console.WriteLine("Hello, World!");
	\end{lstlisting}
	
	\subsection{Tipe data}
	\begin{lstlisting}
	int var = 23;
	decimal var = 232000;
	double var = 0.2;
	float var = 0.2;
	string var = "Ini string";
	char var = 'o';
	bool var = false;
	DateTime var = DateTime.Parse("2023-12-27 20:30:11");
	\end{lstlisting}
	
	\subsection{Manipulasi string}
	\begin{lstlisting}
	string namaDepan = "Rizky";
	string namaBelakang = "Denianto";
	string namaPanjang = namaDepan + ' ' + namaBelakang;
	
	StringBuilder stringBuilder = new StringBuilder();
	stringBuilder.Append(namaDepan);
	stringBuilder.Append(' ');
	stringBuilder.Append(namaBelakang);
	namaPanjang = stringBuilder.ToString();
	
	namaPanjang = string.Format("{0} {1}", namaDepan, namaBelakang);
	
	namaPanjang = $"{namaDepan} {namaBelakang}";
	
	int panjangString = namaPanjang.Length;
	string trimString = namaPanjang.Trim();
	string potongString = namaPanjang.Substring(0, 4);
	string kapitalString = namaPanjang.ToUpper();
	string kecilString = namaPanjang.ToLower();
	int indeksKarakter = namaPanjang.IndexOf('r');
	\end{lstlisting}
	
	\subsection{Manipulasi tanggal}
	\begin{lstlisting}
	DateTime waktudanTanggal = DateTime.Now;
	DateTime minimal = DateTime.MinValue;
	DateTime maksimal = DateTime.MaxValue;
	
	int tahun = waktudanTanggal.Year;
	int bulan = waktudanTanggal.Month;
	int hari = waktudanTanggal.Day;
	int jam = waktudanTanggal.Hour;
	
	waktudanTanggal = DateTime.Parse("2023-12-05 13:30:00");
	string stringWaktudanTanggal = waktudanTanggal.ToString(
	"dd MM yyyy HH:mm:ss",
	CultureInfo.CreateSpecificCulture("id-ID")
	);
	
	DateTime semingguLagi = waktudanTanggal.AddDays(7);
	
	TimeSpan rentangWaktu = waktudanTanggal - semingguLagi;
	\end{lstlisting}
	
	\subsection{Manipulasi angka}
	\begin{lstlisting}
	decimal harga = 20_000;
	string stringHarga = harga.ToString(
	"C3",
	CultureInfo.CreateSpecificCulture("id-ID")
	);
	
	double diskon = 0.25;
	string stringDiskon = diskon.ToString("P0");
	\end{lstlisting}
	
	\subsection{Tipe data yang dapat \textit{null}}
	\begin{lstlisting}
	int? angka = 0;
	
	bool adaIsiAngka = angka.HasValue;
	int isiAngka = angka.Value; // Tidak bisa dicetak jika isinya null
	int isiAwalAngka = angka.GetValueOrDefault(0);
	\end{lstlisting}

	\subsubsection{Latihan}
	\begin{lstlisting}
	Console.WriteLine("Masukkan operator:");
	string op = Console.ReadLine();
	Console.WriteLine("Masukkan angka pertama:");
	string angka = Console.ReadLine();
	Console.WriteLine("Masukkan angka kedua:");
	string angka2 = Console.ReadLine();
	
	double hasil = double.Parse(angka + op + angka2);
	Console.WriteLine(hasil);
	\end{lstlisting}

	\subsection{Kondisi IF}
	\begin{lstlisting}
	int kondisi = 0;
	
	if (kondisi == 0){
		Console.WriteLine("Nol");
	} else if (kondisi == 1) {
		Console.WriteLine("Satu");
	} else {
		Console.WriteLine("Dua");
	}
	\end{lstlisting}

	\subsection{Switch Case}
	\begin{lstlisting}
	int kondisi = 0;
	
	switch (kondisi)
	{
		case 0:
		Console.WriteLine("Nol");
		break;
		case 1:
		Console.WriteLine("Satu");
		break;
		case 2:
		Console.WriteLine("Dua");
		break;
		default:
		Console.WriteLine("Lainnya");
		break;
	}
	\end{lstlisting}

	\subsection{Seleksi}
	\begin{lstlisting}
	int kondisi = 0;
	string kondisi2 = kondisi == 0 ? "Nol" : "Lainnya";
	\end{lstlisting}

	\subsection{Perulangan FOR}
	\begin{lstlisting}
	int kondisi = 0;
	for (int i = 0; i < 8; i++)
	{
		kondisi++;
		
		for (int j = 0; j < 8; j++){
			if (j % 2 == 0) continue;
			else if (j == 4) break;
			
			kondisi++;
		}
	}
	\end{lstlisting}

	\subsection{Perulangan WHILE}
	\begin{lstlisting}
	int kondisi = 0;
	while (kondisi < 8)
	{
		kondisi++;
		
		Console.WriteLine("While " + kondisi);
	}
	
	do
	{
		kondisi++;
		
		Console.WriteLine("Do While " + kondisi);
	}
	while (kondisi < 16);
	\end{lstlisting}

	\subsubsection{Latihan}
	\begin{lstlisting}
	while (true)
	{
		Console.WriteLine("Masukkan kalimat:");
		string kalimat = Console.ReadLine();
		bool valid = true;
		
		int pdKalimat = kalimat.Length;
		for (int i = 0; i < pdKalimat; i++)
		{
			if (Int32.TryParse(kalimat[i].ToString(), out int _))
			{
				valid = false;
				break;
			}
		}
		
		if (valid)
		{
			Console.WriteLine("================");
			Console.WriteLine("Camel Case");
			string camelCase = "";
			bool kapital = false;
			for (int i = 0; i < pdKalimat; i++)
			{
				if (kalimat[i] != ' ') kapital = false;
				else
				{
					kapital = true;
					i++;
				}
				
				if (!kapital) camelCase += kalimat[i];
				else camelCase += kalimat[i].ToString().ToUpper();
				
			}
			Console.WriteLine(camelCase);
			
			Console.WriteLine("Pascal Case");
			string pascalCase = "";
			bool awal = true;
			kapital = false;
			for (int i = 0; i < pdKalimat; i++)
			{
				if (kalimat[i] != ' ') kapital = false;
				else
				{
					kapital = true;
					i++;
				}
				
				if (!kapital && !awal) pascalCase += kalimat[i];
				else
				{
					pascalCase += kalimat[i].ToString().ToUpper();
					awal = false;
				}
				
			}
			Console.WriteLine(pascalCase);
			
			Console.WriteLine("Snake Case");
			string snakeCase = "";
			for (int i = 0; i < pdKalimat; i++)
			{
				if (kalimat[i] == ' ')
				{
					snakeCase += '_';
					continue;
				}
				
				snakeCase += kalimat[i];
				
			}
			Console.WriteLine(snakeCase + '\n');
		}
	}
	\end{lstlisting}
	
	\subsubsection{Latihan}
	\begin{lstlisting}
	while (true)
	{
		Console.WriteLine("Masukkan kalimat:");
		string kalimat = Console.ReadLine();
		bool valid = true;
		
		int pdKalimat = kalimat.Length;
		for (int i = 0; i < pdKalimat; i++)
		{
			if (Int32.TryParse(kalimat[i].ToString(), out int _))
			{
				valid = false;
				break;
			}
		}
		
		if (valid)
		{
			string hurufKonsonan = "";
			string hurufVokal = "";
			
			for (int i = 0; i < pdKalimat; i++)
			{
				if (kalimat[i] == ' ')
				{
					hurufKonsonan += ' ';
					hurufVokal += ' ';
				}
				else
				{
					if (
					kalimat[i].ToString().Contains('a')
					|| kalimat[i].ToString().Contains('i')
					|| kalimat[i].ToString().Contains('u')
					|| kalimat[i].ToString().Contains('e')
					|| kalimat[i].ToString().Contains('o')
					)
					{
						hurufVokal += kalimat[i];
					}
					else
					{
						hurufKonsonan += kalimat[i];
					}
				}
			}
			Console.WriteLine("================");
			Console.WriteLine("Huruf konsonan:");
			Console.WriteLine(hurufKonsonan);
			Console.WriteLine("Huruf vokal:");
			Console.WriteLine(hurufVokal + '\n');
		}
	}
	\end{lstlisting}
	
	\subsection{Susunan/Array Satu Dimensi)}
	\begin{lstlisting}
	int[] susunan = new int[8];
	int[] susunan = new int[]{
		0,
		1,
		2,
		3,
		4
	};
	int[] susunan = {
		0,
		1,
		2,
		3,
		4
	};
	
	susunan[5] = 0;
	
	Console.WriteLine(susunan.Length);
	\end{lstlisting}
	
	\subsection{Susunan/Array Banyak Dimensi}
	\begin{lstlisting}
	int[,] susunan = new int[2, 3];
	int[,] susunan = {{0, 1, 2}, {2, 3, 4}, {7, 8, 9}};
	object[,] susunan = { { "1", 2, 3, "C" }, { "Aku", 5, 6, "B" }, { "Kamu", 8, 9, "A" } };
	
	int dimensiSusunan = susunan.Rank;
	Console.WriteLine(dimensiSusunan);
	
	foreach (object o in susunan) Console.WriteLine(o);
	\end{lstlisting}
	
	\subsection{Susunan/Array Bergerigi \textit{(Jagged Array)}}
	\begin{lstlisting}
	int[][] susunan = new int[2][];
	susunan[0] = new int[2];
	susunan[0][0] = 0;
	susunan[0][1] = 0;
	susunan[1] = new int[2];
	susunan[1][0] = 0;
	susunan[1][1] = 0;
	
	int[][] susunan = new int[3][]{
		new int[2]{0, 2},
		new int[1]{2},
		new int[3]{0, 8, 9}
	};
	
	foreach (int i in susunan[0]) Console.WriteLine(i);
	foreach (int i in susunan[1]) Console.WriteLine(i);
	\end{lstlisting}
	
	\subsection{List}
	\begin{lstlisting}
	List<string> daftar = new List<string>();
	daftar.Add("Harimau");
	daftar.Add("Singa");
	daftar.Add("Kucing");
	daftar.Insert(0, "Kucing");
	daftar.Insert(4, "Kucing");
	daftar.Remove("Harimau");
	daftar.RemoveAt(0);
	daftar.Reverse();
	daftar.Clear();
	// daftar.RemoveAll();
	daftar.AddRange(daftar); // Gabung List<>
	
	List<string> daftar = new List<string>(){
		"Harimau",
		"Singa",
		"Kucing"
	};
	
	List<string[]> daftar = new List<string[]>(){
		new string[]{
			"Harimau",
			"Ikan",
			"Kura-Kura",
			"Kupu-Kupu",
			"Undur-Undur"
		},
		new string[]{
			"Mawar",
			"Melati",
			"Poon",
			"Terompet",
			"Matahari"
		}
	};
	
	Console.WriteLine(daftar.Count);
	
	foreach (string d in daftar) Console.WriteLine(d);
	\end{lstlisting}
	
	\subsection{ArrayList}
	\begin{lstlisting}
	ArrayList daftar = new ArrayList();
	daftar.Add("Banyak");
	daftar.Add(324);
	daftar.Add(false);
	daftar.Insert(0, 20_000);
	
	Console.WriteLine(daftar.IsFixedSize);
	foreach(object d in daftar) Console.WriteLine(d);
	\end{lstlisting}
	
	\subsection{Dictionary}
	\begin{lstlisting}
	Dictionary<int, string> daftar = new Dictionary<int, string>();
	daftar.Add(1, "Harimau");
	daftar.Add(2, "Singa");
	daftar.Add(3, "Macan");
	
	foreach (KeyValuePair<int, string> d in daftar)
	{
		Console.WriteLine(d.Key);
		Console.WriteLine(d.Value);
	}
	foreach (var d in daftar)
	{
		Console.WriteLine(d.Key);
		Console.WriteLine(d.Value);
	}
	\end{lstlisting}
	
	\subsection{Fungsi/Metode \textit{(Method)}}
	\begin{lstlisting}
	public static void FungsiVoid(string teks, string[] susunanTeks, params string[] susunanTeks2)
	{
		Console.WriteLine("Ini fungsi " + teks);
		foreach (string t in susunanTeks) Console.WriteLine("Ini fungsi " + t);
		foreach (string t in susunanTeks2) Console.WriteLine("Ini fungsi " + t);
	}
	
	public static void FungsiVoid2(string varTidakWajib = "tidak wajib diisi")
	{
		Console.WriteLine(varTidakWajib);
	}
	
	public static string FungsiString(string teks)
	{
		return teks + teks;
	}
	
	public static string FungsiEstafet()
	{
		return FungsiString("Fungsi estafet dijalankan ");
	}
	
	public static void FungsiRekursif(int i = 0)
	{
		i++;
		Console.WriteLine(i);
		
		if (i >= 100) return;
		else FungsiRekursif(i);
	}
	\end{lstlisting}
	
	\subsubsection{Latihan}
	\begin{lstlisting}
	public static void Menu(string[] daftarSnack, string[] daftarMinuman, int[] hasilPilihan)
	{
		if (hasilPilihan[0] != 0 && hasilPilihan[1] != 0)
		{
			Console.WriteLine($"Snack dan minuman yang Anda pilih adalah {daftarSnack[hasilPilihan[0] - 1]} dan {daftarMinuman[hasilPilihan[1] - 1]}");
			
			Console.WriteLine("Pilih ulang? (y/t)");
			string pilihUlang = Console.ReadLine();
			if (pilihUlang.ToLower() == "t") Console.WriteLine("Kelar");
			else if (pilihUlang.ToLower() == "y") Menu(daftarSnack, daftarMinuman, new int[] { 0, 0 });
			else
			{
				Console.WriteLine("Pilihan tidak valid\n");
				Menu(daftarSnack, daftarMinuman, hasilPilihan);
			}
		}
		else
		{
			try
			{
				Console.WriteLine("Pilih Menu:");
				Console.WriteLine("1. Snack");
				Console.WriteLine("2. Minuman");
				int pilihan = Int32.Parse(Console.ReadLine());
				
				if (pilihan <= 0 || pilihan > 2)
				{
					Console.WriteLine("Pilihan tidak valid\n");
					Menu(daftarSnack, daftarMinuman, hasilPilihan);
				}
				else
				{
					if (pilihan == 1)
					{
						int n = Snack(daftarSnack);
						hasilPilihan[0] = n != 0 ? n : hasilPilihan[0];
					}
					else if (pilihan == 2)
					{
						int n = Minuman(daftarMinuman);
						hasilPilihan[1] = n != 0 ? n : hasilPilihan[1];
					}
					Menu(daftarSnack, daftarMinuman, hasilPilihan);
				}
			}
			catch
			{
				Console.WriteLine("Pilihan tidak valid\n");
				Menu(daftarSnack, daftarMinuman, hasilPilihan);
			}
		}
	}
	
	public static int Snack(string[] daftarSnack)
	{
		try
		{
			Console.WriteLine("Pilih Snack:");
			for (int i = 0; i < daftarSnack.Length; i++) Console.WriteLine($"{i + 1}. {daftarSnack[i]}");
			int pilihan = Int32.Parse(Console.ReadLine());
			
			if (pilihan > 0 && pilihan < daftarSnack.Length)
			{
				Console.WriteLine($"Snack yang anda pilih adalah {daftarSnack[pilihan - 1]}\n");
				return pilihan;
			}
			else
			{
				if (pilihan != daftarSnack.Length)
				{
					Console.WriteLine("Pilihan tidak valid\n");
					Snack(daftarSnack);
				}
				return 0;
			}
		}
		catch
		{
			Console.WriteLine("Pilihan tidak valid\n");
			Snack(daftarSnack);
			return 0;
		}
	}
	
	public static int Minuman(string[] daftarMinuman)
	{
		try
		{
			Console.WriteLine("Pilih minuman:");
			for (int i = 0; i < daftarMinuman.Length; i++) Console.WriteLine($"{i + 1}. {daftarMinuman[i]}");
			int pilihan = Int32.Parse(Console.ReadLine());
			
			if (pilihan > 0 && pilihan < daftarMinuman.Length)
			{
				Console.WriteLine($"Minuman yang anda pilih adalah {daftarMinuman[pilihan - 1]}\n");
				return pilihan;
			}
			else
			{
				if (pilihan != daftarMinuman.Length)
				{
					Console.WriteLine("Pilihan tidak valid\n");
					Minuman(daftarMinuman);
				}
				return 0;
			}
		}
		catch
		{
			Console.WriteLine("Pilihan tidak valid\n");
			Minuman(daftarMinuman);
			return 0;
		}
	}
	\end{lstlisting}
	
	\subsection{Fungsi/Metode Overloading}
	\begin{lstlisting}
	public static string Fungsi()
	{
		return "";
	}
	public static string Fungsi(string a)
	{
		return "";
	}
	public static int Fungsi(int a)
	{
		return a;
	}
	public static decimal Fungsi(decimal a)
	{
		return a;
	}
	public static int Fungsi(int a, int b)
	{
		return a + b / a - b;
	}
	\end{lstlisting}
	
	\subsection{Dasar OOP}
	\begin{lstlisting}
	namespace ChapterFive;
	
	public class Karyawan
	{
		public string Nama = "";
		
		private string _kelamin = "";
		public string Kelamin
		{
			get => _kelamin;
			set => _kelamin = value;
		}
		
		
		public string Alamat { get; set; } = "";
		
		public string Alamat2 { get; } = "";
		public Karyawan(string alamat, string alamat2)
		{
			this.Alamat = alamat;
			this.Alamat2 = alamat2;
		}
		public Karyawan(string alamat2)
		{
			this.Alamat2 = alamat2;
		}
		
		
		private DateTime _lahir;
		
		public DateTime Lahir
		{
			get
			{
				return this._lahir;
			}
			set
			{
				this._lahir = value;
			}
		}
		
		
		public string AmbilNama()
		{
			return this.Nama;
		}
		
		public void AturNama(string nama)
		{
			this.Nama = nama;
		}
	}
	\end{lstlisting}

\end{document}